\documentclass[a4paper,11pt]{article}

\usepackage{xltxtra}
\usepackage{fontspec}
\usepackage{xunicode}
\usepackage{url}
\usepackage{graphicx}
\usepackage{multicol}
\usepackage{color}
\usepackage{xgreek}
\usepackage{fancyhdr}
\usepackage{hyperref}
\usepackage{tikz}
\usepackage{bbding}

\pagestyle{fancy}

\defaultfontfeatures{Mapping=tex-text}
\setromanfont{Times}
\setsansfont{Epigrafica}
\setmonofont[Scale=0.8]{Courier}

\title{
	Πλήρης ιδιωτικότητα σε δημοπρασίες \\
	\vspace{3mm}
	\normalsize{\textbf{How to obtain full privacy in auctions} by \textbf{Felix Brandt}} \\
}
\author{
	Ιωάννης Καλαντζής\\
	Α.Μ.: 3928\\
	\texttt{ikalantzis@ceid.upatras.gr}
\and
	Ηλίας-Δημήτρης Βραχνής\\
	Α.Μ.: 3899\\
	\texttt{vrachnis@ceid.upatras.gr}
\and
	Μιχαήλ-Άγγελος Σίμος\\
	Α.Μ.: 4015\\
	\texttt{asimos@ceid.upatras.gr}
}
\date{}

\begin{document}

\maketitle

\begin{abstract} Το θέμα με το οποίο καταπιάνεται το paper που καλούμαστε να σχολιάσουμε είναι η ασφάλεια
δημοπρασιών. Αναγνωρίζοντας την αυξανόμενη σημασία της ιδιωτικότητας των συναλλαγών στο σχεδιασμό δημοπρασιών,
προτείνονται τεχνικές και πρωτόκολλα δημοπρασιών πρώτης τιμής καθώς και (ν+1)-ιοστής τιμής που αποκαλύπτουν
μόνο την ταυτότητα των "νικητών" και την τιμή πώλησης. Επιπλέον, αν είναι επιθυμητό, δεν παρέχεται καμία
επιπλέον πληροφορία στους "χαμένους" πλειοδότες πέραν της ήττας τους. Το προτεινόμενο μοντέλο ασφαλείας
βασίζεται στην υπολογιστική δυσκολία εντοπισμού (computational intractability) και δεν επικαλείται την
αξιοπιστία άλλων συμβαλλόμενων προσώπων (π.χ. ο δημοπράτης).

Παρουσιάζεται επίσης και μια υλοποίηση των προτεινόμενων τεχνικών βασιζόμενη στο σύστημα κρυπτογράφησης El
Gamal. Τα απορρέοντα πρωτόκολλα απαιτούν μονάχα τρείς γύρους μετάδοσης μεταξύ των πλειοδοτών. Η πολυπλοκότητα
επικοινωνίας είναι γραμμική στον αριθμό των πιθανών προσφορών. \end{abstract}

\section{Introduction} Τα τελευταία χρόνια, οι δημοπρασίες έχουν εξελιχθεί σε βασικό άξονα του ηλεκτρονικού
εμπορίου. Δεν αποτελούν μόνο μηχανισμούς για την πώληση αγαθών αλλά έχουν εφαρμογή και σε φαινομενικά άσχετους
τομείς όπως η ανάθεση έργων, ο προγραμματισμός δραστηριοτήτων ή ακόμα και η εύρεση του συντομότερου μονοπατιού
σε ένα δίκτυο. Ταυτόχρονα όμως, η ανάγκη για ύπαρξη ιδιωτικότητας στις δημοπρασίες έχει εξελιχθεί σε παράγοντα
μείζονος σημασίας. Τα κρυπτογραφικά πρωτόκολλα που παρουσιάζονται στο paper τους Brandt διαφέρουν από
προηγούμενες έρευνες που έχουν γίνει πάνω στο θέμα στο γεγονός ότι δεν απαιτούν αξιόπιστους τρίτους (trusted
third-parties). Βασίζονται μονάχα σε υπολογιστική δυσκολία εντοπισμού (computational intractability). Η
ιδιωτικότητα εξασφαλίζεται στο μέγιστο βαθμό χωρίς να γίνεται συμβιβασμός στην αποδοτικότητα των γύρων.

Το σκηνικό μας αποτελείται από έναν πωλητή και ν πλειοδότες με σκοπό να καταλήξουν σε κάποια συμφωνία για την
πώληση ενός αγαθού. Μιλάμε πάντα για κλειστές δημοπρασίες (sealed-bid auctions) που σημαίνει ότι ο κάθε
πλειοδότης γνωρίζει μόνο τη δική του προσφορά και καμία άλλη. Οι δύο βασικοί μηχανισμοί που οδηγούν σε
συμφωνία είναι οι \textbf{\emph{δημοπρασίες πρώτης τιμής}} και οι δημοπρασίες \textbf{\emph{δεύτερης τιμής
(Vickrey auctions)}}.

Και στους δύο παραπάνω μηχανισμούς ο κάθε πλειοδότης υποβάλει μια κρυφή προσφορά στον δημοπράτη δηλώνοντας το
ποσό που είναι διατεθειμένος να πληρώσει. Ο δημοπράτης ορίζει ως νικητή τον πλειοδότη με την υψηλότερη
προσφορά. Στις δημοπρασίες πρώτης τιμής, ο νικητής πληρώνει το ποσό που προσέφερε ο ίδιος ενώ στις δημοπρασίες
Vickrey καλείται να πληρώσει το ποσό της δεύτερης μεγαλύτερης προσφοράς.

Και οι δύο μηχανισμοί έχουν δυνατά και αδύνατα σημεία. Για παράδειγμα, οι δημοπρασίες πρώτης τιμής έχουν
καλύτερα έσοδα όταν οι πλειοδότες είναι περισσότερο συντηρητικοί ενώ οι δημοπρασίες Vickrey είναι
strategyproof (όρος από τη θεωρία παιγνίων) που σημαίνει ότι είναι πάντα προτιμότερο για τους πλειοδότες να
προσφέρουν με βάση την πραγματική τους εκτίμηση για την αξία του προς πώλησην αγαθού. Οι ανεπιθύμητες
ενέργειες του πλεονεκτήματος αυτού συνεισφέρουν στο γεγονόε οτι οι δημοπρασίες Vickrey έχουν μικρή πρακτική
εφαρμογή για τους εξής λόγους:

\begin{itemize}
	\item Οι πλειοδότες είναι απρόθυμοι να αποκαλύψουν την πραγματική τους εκτίμηση στον πλειοδότη. 
	\item Οι πλειοδότες αμφισβητούν την ακρίβεια της δημοπρασίας αφού δεν πληρώνουν το ποσό που προσέφεραν.
\end{itemize}

Ένα κλασσικό παράδειγμα που υποδεικνύει τα παραπάνω είναι όταν ο δημοπράτης κατασκευάζει μια πλαστή δεύτερη
μεγαλύτερη προσφορά με σκοπό να αυξήσει τα έσοδα του. Και τα δύο παραπάνω ζητήματα βασίζονται στην έλλειψη
εμπιστοσύνης στον δημοπράτη. Για αυτό το λόγο, θα ήταν επιθυμητό με κάποιο τρόπο να "εξαναγκάσουμε" τον
δημοπράτη να επιλέγει πάντα την σωστή έκβαση και να "απαγορεύσουμε" τη διάδοση πληροφοριών των ιδιωτικών
προσφορών. Τα τελευταία χρόνια, έχουν προταθεί διάφορα σχέδια για την επίτευξη αυτού του σκοπού. Πρακτικά όμως
όλα βασίζονται σε αξιόπιστα συμβαλλόμενα μέρη. Στόχος του paper του Brandt είναι η κατασκευή αποδοτικών
πολυκομματικών πρωτοκόλλων που επιτρέπουν στους \emph{πλειοδότες} να λογαριάσουν απο κοινού την έκβαση της
δημοπρασίας χωρίς να αποκαλύψουν περαιτέρω πληροφορίες.

Τα πρωτόκολλα που εισάγει το paper αφορούν δημοπρασίες πρώτης τιμής και (ν+1)-ιοστής τιμής. Οι τελευταίες
είναι γενίκευση των δημοπρασιών δεύτερης τιμής.

\subsection{Δομή του paper}

Η δομή του paper έχει ως εξής. Αρχικά, περιγράφει το υποκείμενο γενικό μοντέλο ασφαλείας. Ακολουθεί μια αναφορά
σε σχετικές προηγούμενες έρευνες πάνω σε πρωτόκολλα δημοπρασιών και σύγκριση με την προτεινόμενη προσέγγιση. Το
επόμενο κομμάτι περιέχει μια αναλυτική περιγραφή των εννοιών που χρησιμοποιούνται στην υλοποίηση που ακολουθεί.
Έπεται μια ανάλυση πάνω στην ασφάλεια και την αποδοτικότητα των προτεινόμενων πρωτοκόλλων. Στο τέλος, ο
συντάκτης του paper κάνει μια σύντομη ανασκόπηση των αποτελεσμάτων.

\section{Security Model}

Πρωταρχικός στόχος του ερευνητή είναι ιδιωτικότητα στις δημοπρασίες που δεν μπορεί να παραβιαστεί από κανένα
συνασπισμό τρίτων προσώπων ή πλειοδοτών. Για το λόγο αυτό, συνηγορεί σε ένα μοντέλο ασφαλείας στο οποίο οι
πλειοδότες λογαριάζουν από κοινού την έκβαση της δημοπρασίας με τέτοιο τρόπο ώστε κανένα υποσύνολο τους να μην
μπορεί να αποκαλύψει ιδιωτικές πληροφορίες. Παρά το γεγονός ότι είναι ανεπιθύμητη η εκτεταμένη αλληλεπίδραση με
τους πλειοδότες, ο ερευνητής δεν μπορεί να το αποφύγει και προσπαθεί να ελαχιστοποιήσει την επίδρασή του
κρατώντας την πολυπλοκότητα των γύρων στο ελάχιστο, γεγονός που απαιτεί την ύπαρξη ενός καναλιού εκπομπής. Τα
μειονεκτήματα που παρουσιάζονται λόγω της συγκεκριμένης προσέγγισης, είναι η μικρή ανοχή της καθώς και σχετικά
υψηλή υπολογιστική και επικοινωνιακή πολυπλοκότητα. Είναι γεγονός ότι σε πλειστηριασμούς που απαιτούν τόσο
μεγάλο βαθμό ιδιωτικότητας συνήθως γίνονται με λίγους (και γνωστούς) πλειοδότες.

Ο ερευνητής χρησιμοποιεί κρυπτογραφικά πρωτόκολλα για n πλειοδότες και έναν πωλητή (στη συνέχεια θα λέγονται
συντελεστές). Κάθε πλειοδότης i κατέχει μια προσωπική είσοδο, την προσφορά του bid[i] E B. Οι συντελεστές
συμπλέκονται σε ένα πρωτόκολλο πολλών ατόμων ώστε κοινά και με ασφάλεια να αποφασίσουν το αποτέλεσμα της
συνάρτησης f. Στη δική μας σκοπιά, η ασφάλεια αποτελείται από ορθότητα και ιδιωτικότητα. Τα συνήθη
αποτελέσματα από ασφαλείς υπολογισμούς πολλών προσώπων υποδεικνύουν ότι κάθε συνάρτηση f μπορεί να υπολογιστεί
με ασφάλεια όταν
\begin{itemize}
	\item το πολύ $\lfloor \frac{n-1}{2} \rfloor$ συντελεστές μοιράζονται τις πληροφορίες τους και υπάρχουν παραλαγές κατωφλιού, ή 
	\item το πολύ $\lfloor \frac{n-1}{3} \rfloor$ συντελεστές μοιράζονται τις πληροφορίες τους και υπάρχει πλήρες δίκτυο από
ιδιωτικά κανάλια.
\end{itemize}

Η πρώτη υπόθεση είναι γνωστή ως υπολογιστικό μοντέλο ενώ η δεύτερη είναι γνωστή ως το άνευ όρων μοντέλο.
Ωστόσο, καμία υπόθεση από τις δυο δεν μπορεί να γίνει στην δική μας περίπτωση γιατί θα επέτρεπε σε υποσύνολα
πλειοδοτών να προκαθορίσουν το αποτέλεσμα του πλειστηριασμού και να ανακαλύψουν τις προσφορές των άλλων
συντελεστών. Παρ' όλα αυτά, υπάρχουν επιπλέον υποθέσεις που επιτρέπουν τον ασφαλή υπολογισμό αυθαίρετων
συναρτήσεων στο υπολογιστικό μοντέλο, και ένα περιορισμένο σύνολο συναρτήσεων στο άνευ όρων μοντέλο, χωρίς
έμπιστα κατώφλια στους συντελεστές. Ιδιωτικότητα που στηρίζεται στο γεγονός ότι δεν είναι όλοι οι συντελεστές
συνεννοημένοι θα αναφέρεται ως πλήρης ιδιωτικότητα. Στο υπολογιστικό μοντέλο, κάθε συνάρτηση f μπορεί να
υπολογιστεί πλήρως κατ'ιδίαν όταν 
\begin{itemize}
	\item υπάρχουν παραλαγές κατωφλιού, και 
	\item ένας καθορισμένος συντελεστής δεν παραιτείται ή αποκαλύπτει πληροφορίες πρόωρα.
\end{itemize}

Στα πρωτόκολλα πλειστηριασμών που παρουσιάζονται σε αυτό το paper, ο πωλητής θα πάρει το ρόλο του καθορισμένου συντελεστή. Είναι σημαντικό να σημειώσουμε ότι ακόμα και στην περίπτωση που ο πωλητής παραιτείται ή αποκαλύπτει πληροφορίες πρόωρα, το χειρότερο που θα μπορούσε να συμβεί είναι ότι ένας πλειοδότης μαθαίνει το αποτέλεσμα και παραιτείται πριν οι υπόλοιποι συντελεστές μάθουν το αποτέλεσμα. Η ιδιωτικότητα των προσφορών δεν επηρεάζεται από πρόωρες παραιτήσεις. Ένας άλλος κοινός τρόπος για να επιτύχουμε δικαιοσύνη χωρίς έμπιστη πλειοψηφία είναι η αυξανόμενη αποκάλυψη μυστικών.

Κάθε φορά που ένας κακόβουλος πλειοδότης παρεμποδίζει το πρωτόκολλο στέλνοντας εσφαλμένα μηνύματα ή αδυνατώντας να αποδείξει την ορθότητα της συμπεριφοράς του στη μηδαμινή γνώση, αυτός ο πλειοδότης αφαιρείται και το πρωτόκολλο επανεκκινείται. Θεωρούμε ότι το “κοινό” παρατηρεί το πρωτόκολλο και γι’ αυτό ένας κακόβουλος χρήστης μπορεί αδιαμφισβήτητα να ταυτοποιηθεί, ανεξάρτητα από το πόσοι από τους υπόλοιπους συντελεστές είναι αξιόπιστοι. Επειδή στους κακόβουλους χρήστες μπορεί εύκολα να κοπεί πρόστιμο και δεν κερδίζουν καθόλου πληροφορίες, δε θα έπρεπε να υπάρχει κίνητρο για παρεμπόδιση του πλειστηριασμού και γι’ αυτό από εδώ και στο εξής θεωρούμε ότι μία εκτέλεση του πρωτοκόλλου αρκεί.

Εξαιτίας της ανεπάρκειας των υπαρχόντων MPC schemes, είναι αναπόφευκτο να σχεδιάσουμε πρωτόκολλα ειδικού σκοπού για τον υπολογισμό συγκεκριμένων συναρτήσεων. Αν $b = (bid_1, bid_2, ..., bid_n)$ είναι το διάνυσμα με όλες τις προσφορές και $f: B^n -> O^(n+1)$ η συνάρτηση εξόδου όπου $f(b) = (f_1(b), f_2(b), ..., f_n(b),(f_1(b), f_2(b), ...,f_n(b)))$ ώστε ο πλειοδότης $i$ μαθαίνει $f_i(b)$ και ο πωλητής μαθαίνει $(f_1(b), f_2(b), ..., f_n(b))$. Αν ο πλειοδότης $i$ κερδίσει τον πλειστηριασμό, το $f_i(b)$ παράγει τη τιμή πώλησης. Αλλιώς “άχρηστες” πληροφορίες επιστρέφονται. Αυτό θα λέγεται σκηνικό ιδιωτικού αποτελέσματος αφού το αποτέλεσμα ανακοινώνεται μόνο στα ενδιαφερόμενα μέρη. Για λόγους διαφάνειας και αποτελεσματικότητας, θα θεωρήσουμε επίσης τον υπολογισμό μιας συνάρτησης δημοσίου αποτελέσματος όπου όλα τα $f_i(b)$ είναι πανομοιότυπα και παράγουν την ταυτότητα του νικητή του πλειστηριασμού και τη τιμή πώλησης.

\section{Related Work}

Το ενδιαφέρον σε κρυπτογραφικά πρωτόκολλα για δημοπρασίες έχει αυξηθεί δραματικά. Έχουν προταθεί διάφορα
ασφαλή μοντέλα για διεξαγωγή δημοπρασιών σφραγισμένης προσφοράς (sealed-bid auctions). Ένα κοινό στοιχείο που
μοιράζονται όλα τα υπάρχοντα πρωτόκολλα είναι ότι η ιδιωτικότητα λαμβάνεται κατανέμοντας τον υπολογισμό της
έκβασης της δημοπρασίας σε μια ομάδα τρίτων.

Υπάρχουν συμμετρικά πρωτόκολλα, όπου υπάρχουν πολλοί δημοπράτες που αποφασίζουν από κοινού την έκβαση
χρησιμοποιώντας threshold MPC, και συμμετρικά πρωτόκολλα, τα οποία εισάγουν μια επιπλέον αρχή στη δημοπρασίας
(auction issuer or auction authority).

	\subsection{Cryptographic auction protocols}
	\subsection{Bidder-resolved auctions}
	
\section{Protocol description}

	\subsection{First-price auctions}
	\subsection{(M+1)st-price auctions}

\section{Protocol implementation}

	\subsection{El Gamal encryption}
	\subsection{Zero-knowledge proofs}

\section{Συμπεράσματα} 
Ο συντάκτης του paper παρουσίασε κρυπτογραφικά πρωτόκολλα σταθετού αριθμού γύρων για διάφορους τύπους
δημοπρασιών κλειστής προσφοράς (sealed-bid auctions). Η ασφάλεια των προτεινόμενων προτοκόλλων βασίζεται στην
υπολογιστική δυσκολία εντοπισμού.

\end{document}
